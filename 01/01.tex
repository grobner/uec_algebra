\documentclass[dvipdfmx,11pt,notheorems]{beamer}

\usepackage{bxdpx-beamer}
\usepackage{pxjahyper}
\usepackage{minijs}%和文用
\renewcommand{\kanjifamilydefault}{\gtdefault}%和文用


\usetheme{Madrid}
\usefonttheme{professionalfonts}
\setbeamertemplate{frametitle}[default][center]
\setbeamertemplate{navigation symbols}{}
\setbeamercovered{transparent}%好みに応じてどうぞ)
\setbeamertemplate{footline}[page number]
\setbeamerfont{footline}{size=\normalsize,series=\bfseries}
\setbeamercolor{footline}{fg=black,bg=black}
%%%%

\usepackage{amsmath,amssymb}
\usepackage{bm}
\usepackage{graphicx}
\usepackage{ascmac}
\usepackage{mathtools}

\newcommand{\divergence}{\mathrm{div}\,}  %ダイバージェンス
\newcommand{\grad}{\mathrm{grad}\,}  %グラディエント
\newcommand{\rot}{\mathrm{rot}\,}  %ローテーション
\newcommand{\Ker}{\operatorname{Ker}} %核
\newcommand{\Image}{\operatorname{Im}} %像


\title{UEC代数勉強会01回目}
\author{bokuroro}
\date{\today}
\begin{document}
\maketitle
\begin{frame}{群の定義}
  \begin{block}{群の定義}
  \begin{enumerate}
   \item 結合法則 $\forall a,b,c \in G$に対し、$(a\circ b)\circ c=a\circ(b\circ c)$
   \item 単位元の存在 $\exists e \in G$  s.t.  $\forall a \in G$に対し、$a\circ e=e\circ a=a$\\
   このような$e$を\alert{単位元}と呼ぶ.
   \item 逆元の存在 $\forall a \in G$に対し、$\exists b \in G$  s.t.  $a\circ b = b\circ a=e$\\
   このような$b$を$a$の\alert{逆元}と呼び、$a^{-1}$で表す.
  \end{enumerate}
\end{block}
  \alert{二項演算}が定義されていることが前提となっている.\\
  また、交換則$a\circ b=b\circ a$が成り立つものを\alert{可換群}または\alert{Abel群}と呼ぶ.
\end{frame}
\begin{frame}{群の例}
  \begin{exampleblock}{変換群}
    定義は参考書参照\\
    写像$f,g,h \in G: X \rightarrow X$を考えたとき、$x \in X$として、
    \begin{eqnarray*}
      ((h\circ g)\circ f)(x) &=& (h\circ g)(f(x))\\
      &=&(h\circ g)(f(x))\\
      &=&h(g(f(x)))\\
      &=& h((g\circ f)(x))\\
      &=& (h\circ (g\circ f))(x)
    \end{eqnarray*}
    よって結合則成立. \\
    まぁこんなの考えなくてもほぼ自明ですが. \\
    恒等写像$\mathrm{id}(x)=x$が存在すること、逆元が存在することは、写像が全単射より自明ですね.
  \end{exampleblock}
\end{frame}
\begin{frame}{群の例}
  \begin{exampleblock}{対称群}
    要するに$n$個のものの置換の群である. \\
    \begin{equation*}
      \begin{pmatrix}
        1 & 2 & 3 \\
        2 & 1 & 3 \\
      \end{pmatrix}
    \end{equation*}
    と書いたら、対称群$S_3$の元で$1\rightarrow 2,2\rightarrow1,3\rightarrow3$に置き換わることを表す. \\
    \alert{巡回置換}、$(1,2,3)$と互換については参考書参照. 任意の置換は互いに素(交わらない)巡回置換に分解でき、いくつかの互換に分解できる. \\
    積についてはこの先出てくるので演習問題にしておきます.
  \end{exampleblock}
\end{frame}
\begin{frame}{いろいろ定義}
  \begin{block}{定義1.2}
    群の元の総数のことを群の\alert{位数}(order)と呼ぶ.
    \begin{itemize}
      \item 位数が有限 $\rightarrow$ \alert{有限群}
      \item 位数が無限 $\rightarrow$ \alert{無限群}
    \end{itemize}
    また、
    \begin{itemize}
      \item 連続パラメータで依存する $\rightarrow$ \alert{連続群}
      \item それ以外 $\rightarrow$ \alert{離散群}
    \end{itemize}
    と呼ぶ.
  \end{block}
\end{frame}

\begin{frame}{環の定義}
  \begin{block}{環の定義}
    集合$A$に2つの二項演算$(+,\cdot)$が定義されていて、次の性質を持つ.
    \begin{enumerate}
      \item $(A,+)$は可換群をなす.
      \item 乗法$\cdot$は結合法則$(a\cdot b)\cdot c=a\cdot(b\cdot c)$を満たす.
      \item 2つの演算は\alert{分配法則}を満たす.\\
      $\forall a,b,c \in A$に対し、$a\cdot(b+c)=a\cdot b + a\cdot c,    (a+b)\cdot c=a\cdot c+b\cdot c$
    \end{enumerate}
    勘違いしやすいが、必ずしも乗法の単位元を持つ必要はない. \\

  \end{block}
  \begin{exampleblock}{行列環}
    $n$次正方行列の全体が行列の和と積を演算として成り立つ環、行列環$M(n,\mathbb{R})$がある. 加法の単位元はゼロ行列$O$、乗法の単位元は単位行列$E$である.\\
    成分を任意の環としても、再び環となる.
  \end{exampleblock}
\end{frame}

\begin{frame}{零因子}
  \begin{block}{零因子}
    零元と異なる2つの元$x,y$で掛けたもの$xy=0$となってしまうものを\alert{零因子}と呼ぶ. 例えば、二次正方行列の環では
    \begin{equation}
      \begin{pmatrix}
        1 & 0 \\
        0 & 0 \\
      \end{pmatrix}
      \begin{pmatrix}
        0 & 0 \\
        1 & 0 \\
      \end{pmatrix}
      =
      \begin{pmatrix}
        0 & 0 \\
        0 & 0 \\
      \end{pmatrix}
    \end{equation}
    となり、零因子がたくさん存在する.
  \end{block}
\end{frame}

\begin{frame}{体の定義}
  \alert{体}は環の特別なもので、単位可換環$(K,+,\cdot)$において、零元を除いたもの$K^\times := K\backslash \{0\}$が乗法に関して群をなすものを言う. \\
  性質を改めて書けば
  \begin{block}{体の定義}
    \begin{enumerate}
      \item $(K,+)$は可換群をなす.
      \item $(K^\times,\cdot)$は可換群をなす.
      \item $+$と$\cdot$は分配法則で関連する. \\
      $\forall a,b,c \in K$に対し、$a(b+c)=ab + ac,    (a+b)c=ac+bc$
    \end{enumerate}
  \end{block}
\end{frame}

\begin{frame}{有限体}
  \begin{block}{無限体 有限体}
    体に含まれる元の個数が有限個であるものを\alert{有限群}、多くの元を含む体を\alert{無限体}と言う. \\
    無限体の例としては、有理数体$\mathbb{Q}$、実数体$\mathbb{R}$、複素数体$\mathbb{C}$、実係数の有理関数体$\mathbb{R}(x)$、複素係数の有理関数体$\mathbb{C}(x)$が存在する.
  \end{block}
  また、有限体の例としては次である.
  \begin{exampleblock}{有限体$F_p$}
    $p$は素数である.\\
    集合としては、$Z_p$と同じもので、$p$で割ったあまりを並べてある.\\
    $p$が素数の時、$0$以外の元に乗法の逆元が存在することを次ページで証明しておく.
  \end{exampleblock}
\end{frame}
\end{document}
