\documentclass[dvipdfmx,11pt,notheorems]{beamer}

\usepackage{bxdpx-beamer}
\usepackage{pxjahyper}
\usepackage{minijs}%和文用
\renewcommand{\kanjifamilydefault}{\gtdefault}%和文用


\usetheme{Madrid}
\usefonttheme{professionalfonts}
\setbeamertemplate{frametitle}[default][center]
\setbeamertemplate{navigation symbols}{}
\setbeamercovered{transparent}%好みに応じてどうぞ)
\setbeamertemplate{footline}[page number]
\setbeamerfont{footline}{size=\normalsize,series=\bfseries}
\setbeamercolor{footline}{fg=black,bg=black}
%%%%

\usepackage{amsmath,amssymb}
\usepackage{bm}
\usepackage{graphicx}
\usepackage{ascmac}
\usepackage{mathtools}

\newcommand{\divergence}{\mathrm{div}\,}  %ダイバージェンス
\newcommand{\grad}{\mathrm{grad}\,}  %グラディエント
\newcommand{\rot}{\mathrm{rot}\,}  %ローテーション
\newcommand{\Ker}{\operatorname{Ker}} %核
\newcommand{\Image}{\operatorname{Im}} %像


\title{UEC代数勉強会01回目}
\author{bokuroro}
\date{December 26, 2020}
\begin{document}
\maketitle
\begin{frame}{群の定義}
  \begin{block}{群の定義}
  \begin{enumerate}
   \item 結合法則 $\forall a,b,c \in G$に対し、$(a\circ b)\circ c=a\circ(b\circ c)$
   \item 単位元の存在 $\exists e \in G$  s.t.  $\forall a \in G$に対し、$a\circ e=e\circ a=a$\\
   このような$e$を\alert{単位元}と呼ぶ.
   \item 逆元の存在 $\forall a \in G$に対し、$\exists b \in G$  s.t.  $a\circ b = b\circ a=e$\\
   このような$b$を$a$の\alert{逆元}と呼び、$a^{-1}$で表す.
  \end{enumerate}
\end{block}
  \alert{二項演算}が定義されていることが前提となっている.\\
  また、交換則$a\circ b=b\circ a$が成り立つものを\alert{可換群}または\alert{Abel群}と呼ぶ.
\end{frame}
\begin{frame}{群の例}
  \begin{exampleblock}{変換群}
    定義は参考書参照\\
    写像$f,g,h \in G: X \rightarrow X$を考えたとき、$x \in X$として、
    \begin{eqnarray*}
      ((h\circ g)\circ f)(x) &=& (h\circ g)(f(x))\\
      &=&(h\circ g)(f(x))\\
      &=&h(g(f(x)))\\
      &=& h((g\circ f)(x))\\
      &=& (h\circ (g\circ f))(x)
    \end{eqnarray*}
    よって結合則成立. \\
    まぁこんなの考えなくてもほぼ自明ですが. \\
    恒等写像$\mathrm{id}(x)=x$が存在すること、逆元が存在することは、写像が全単射より自明ですね.
  \end{exampleblock}
\end{frame}
\begin{frame}{群の例}
  \begin{exampleblock}{対称群}
    要するに$n$個のものの置換の群である. \\
    \begin{equation*}
      \begin{pmatrix}
        1 & 2 & 3 \\
        2 & 1 & 3 \\
      \end{pmatrix}
    \end{equation*}
    と書いたら、対称群$S_3$の元で$1\rightarrow 2,2\rightarrow1,3\rightarrow3$に置き換わることを表す. \\
    \alert{巡回置換}と互換については参考書参照. 任意の置換は互いに素(交わらない)巡回置換に分解でき、いくつかの互換に分解できる. \\
    積についてはこの先出てくるので演習問題にしておきます.
  \end{exampleblock}
\end{frame}
\begin{frame}{いろいろ定義}
  \begin{block}{定義1.2}
    群の元の総数のことを群の\alert{位数}(order)と呼ぶ.
    \begin{itemize}
      \item 位数が有限 $\rightarrow$ \alert{有限群}
      \item 位数が無限 $\rightarrow$ \alert{無限群}
    \end{itemize}
    また、
    \begin{itemize}
      \item 連続パラメータで依存する $\rightarrow$ \alert{連続群}
      \item それ以外 $\rightarrow$ \alert{離散群}
    \end{itemize}
    と呼ぶ.
  \end{block}
\end{frame}
\begin{frame}{演習問題}
  \begin{exampleblock}{問1.1}
    $S_3$において$p=(1,2,3),q=(1,2)$とおく時、$px=q,yp=q$となる元$x,y$をそれぞれ求めよ.
  \end{exampleblock}
\end{frame}

\begin{frame}{環の定義}
  \begin{block}{環の定義}
    集合$A$に2つの二項演算$(+,\cdot)$が定義されていて、次の性質を持つ.
    \begin{enumerate}
      \item $(A,+)$は可換群をなす.
      \item 乗法$\cdot$は結合法則$(a\cdot b)\cdot c=a\cdot(b\cdot c)$を満たす.
      \item 2つの演算は\alert{分配法則}を満たす.\\
      $\forall a,b,c \in A$に対し、$a\cdot(b+c)=a\cdot b + a\cdot c,    (a+b)\cdot c=a\cdot c+b\cdot c$
    \end{enumerate}
    勘違いしやすいが、必ずしも乗法の単位元を持つ必要はない. \\

  \end{block}
  \begin{exampleblock}{行列環}
    $n$次正方行列の全体が行列の和と積を演算として成り立つ環、行列環$M(n,\mathbb{R})$がある. 加法の単位元はゼロ行列$O$、乗法の単位元は単位行列$E$である.\\
    成分を任意の環としても、再び環となる.
  \end{exampleblock}
\end{frame}

\begin{frame}{零因子}
  \begin{block}{零因子}
    零元と異なる2つの元$x,y$で掛けたもの$xy=0$となってしまうものを\alert{零因子}と呼ぶ. 例えば、二次正方行列の環では
    \begin{equation}
      \begin{pmatrix}
        1 & 0 \\
        0 & 0 \\
      \end{pmatrix}
      \begin{pmatrix}
        0 & 0 \\
        1 & 0 \\
      \end{pmatrix}
      =
      \begin{pmatrix}
        0 & 0 \\
        0 & 0 \\
      \end{pmatrix}
    \end{equation}
    となり、零因子がたくさん存在する.
  \end{block}
  零因子を持つとややこしいので、零因子を持たない環を\alert{整域}と呼び、重宝される.
\end{frame}
\begin{frame}{演習問題}
  \begin{exampleblock}{問1.8}
    $D$を平方因子を持たない整数とする. $a+b\frac{1+\sqrt{D}}{2}$,$a,b\in \mathbb{Z}$の形の複素数の全体が複素数の通常の演算で環となるのは$D$がどのような場合か?
  \end{exampleblock}
\end{frame}
\begin{frame}{体の定義}
  \alert{体}は環の特別なもので、単位可換環$(K,+,\cdot)$において、零元を除いたもの$K^\times := K\backslash \{0\}$が乗法に関して群をなすものを言う. \\
  性質を改めて書けば
  \begin{block}{体の定義}
    \begin{enumerate}
      \item $(K,+)$は可換群をなす.
      \item $(K^\times,\cdot)$は可換群をなす.
      \item $+$と$\cdot$は分配法則で関連する. \\
      $\forall a,b,c \in K$に対し、$a(b+c)=ab + ac,    (a+b)c=ac+bc$
    \end{enumerate}
  \end{block}
\end{frame}

\begin{frame}{有限体}
  \begin{block}{無限体 有限体}
    体に含まれる元の個数が有限個であるものを\alert{有限群}、多くの元を含む体を\alert{無限体}と言う. \\
    無限体の例としては、有理数体$\mathbb{Q}$、実数体$\mathbb{R}$、複素数体$\mathbb{C}$、実係数の有理関数体$\mathbb{R}(x)$、複素係数の有理関数体$\mathbb{C}(x)$が存在する.
  \end{block}
  また、有限体の例としては次である.
  \begin{exampleblock}{有限体$\mathbb{F}_p$}
    $p$は素数である.\\
    集合としては、$\mathbb{Z}_p$と同じもので、$p$で割ったあまりを並べてある.\\
    $p$が素数の時、$0$以外の元に乗法の逆元が存在することを次ページで証明しておく.
  \end{exampleblock}
\end{frame}

\begin{frame}{証明}
  ここでは、下を用いて証明を行います. 直感的には鳩で置き換えた方がわかりやすいですが、適用するときは、こちらの表現を使った方が良いと思います。
  \begin{alertblock}{鳩の巣原理}
    元の個数が等しい二つの有限集合の間に写像があるとき
    \begin{enumerate}
      \item 全射なら単射
      \item 単射なら全射
    \end{enumerate}
  \end{alertblock}
  $1 \le \forall x \le p-1$に対して、乗法の逆元が存在することを証明する.\\
  まず、$p$が素数より、$x$による乗法は$\mathbb{F}_p^\times = \{1,2,\ldots,p-1\}$から、一対一写像を引き起こす.(つまり単射)\\
  なぜなら、$a,b\in \{1,2,\ldots,p-1\}$について$xa=xb$が言えるとき、$p|x(a-b)$で、$|a-b|<p$より、$a=b$が言えるからである.\\

  よって、鳩の巣原理より、この写像は全射であり、$xy=1$となるような$y$が存在する.
\end{frame}
\begin{frame}{補足}
  有限体$\mathbb{F}_p$と、有理数体$\mathbb{Q}$は、これ以上小さな部分体が取れない\alert{素体}(1.6で詳しくやる)というものです.\\
  全ての素体は$\mathbb{Q}$か$\mathbb{F}_p$と同型(全く同じ代数構造を持つこと)であることが言えます.\\
  参考: http://hooktail.sub.jp/algebra/PrimeFiled/
\end{frame}

\begin{frame}{公理を用いた推論}
  全部書いてると冗長になってしまうので、どれも重要ですが,2つだけ書いておきます.
  \begin{alertblock}{命題1.2環の公理の系}
    $\forall a \in A$に対し、$a\cdot 0 = 0 \cdot a=0$、よって$0$は乗法の逆元を持ち得ない. また、零因子も乗法の逆元を持ち得ない.
  \end{alertblock}
  分配法則より、$a\cdot 0=a\cdot (0+0) =a\cdot 0+a\cdot 0$であり、両辺に$a\cdot 0$の加法の逆元$-a\cdot 0$を加えると、結合法則より
  \begin{eqnarray*}
    0 &=& a\cdot0+(-a\cdot0)=(a\cdot0+a\cdot0)+(-a\cdot0)=a\cdot0+(a\cdot0+(-a\cdot0))\\
    &=& a\cdot0+0 = a\cdot0
  \end{eqnarray*}
  また、$0$に逆元$x$があるとすれば、$0=x\cdot 0=1$で矛盾. $x,y\neq0$で、$x\cdot y=0$とし、$x$に乗法の逆元$x^{-1}$があるとすると
  \begin{equation*}
    0 = x^{-1}\cdot 0 = x^{-1}\cdot(x\cdot y)=(x^{-1}\cdot x)\cdot y=1\cdot y = y
  \end{equation*}
  となって矛盾.
\end{frame}
\begin{frame}{公理を用いた推論}
  \begin{alertblock}{命題1.5}
    体には零因子は存在しない. また、これより体の元を係数とする一変数代数方程式$f(x):=a_n x^n+a_1x^{n-1}+\ldots+a_n=0$($a_0\neq0$)は次数$n$より多くの根を持たない.
  \end{alertblock}
  $xy=0$で$x\neq0$なら、体の公理により、$x^{-1}$が存在し、これを両辺にかけると
  \begin{equation*}
    0 = x^{-1}(xy) = (x^{-1}x)y=1\cdot y = y
  \end{equation*}
  となる. また、$f(\alpha)=0$とすると、因数定理
  \begin{equation*}
    f(x) = (x-\alpha)q(x)
  \end{equation*}
  という式が得られ、$q(x)$は$n-1$次だから、数学的帰納法により、主張が証明できる.
\end{frame}
\begin{frame}{演習問題}
  では、いくつか演習問題を解いてみましょう.
  \begin{exampleblock}{問1.12}
    群$G$において、任意の元が$x^2 = e$を満たしていれば$G$は可換群であることを示せ。
  \end{exampleblock}
  \begin{exampleblock}{問1.13}
    群$G$において、二つの元$x,y$が可換であるためには$(xy)^2=x^2y^2$が成立することが必要かつ十分であること証明せよ.
  \end{exampleblock}
\end{frame}
\end{document}
